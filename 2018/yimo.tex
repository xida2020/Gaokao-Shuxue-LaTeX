\documentclass{BHCexam}
\begin{document}
	\biaoti{2018年北京一模}
	\fubiaoti{}
	\maketitle
	\begin{questions}
		\qs 已知$ a,~b $为正实数,则$ “a>1,~b>1” $是$ “\lg a+\lg b>0” $的\xx
		\twoch{充分而不必要条件}{必要而不充分条件}{充分必要条件}{既不充分也不必要条件}
		\qs 下列函数$f(x)$中,其图像上任意一点$ P(x,y) $的坐标都满足条件$ y\le \abs{x} $的函数是\xx
		\onech{$ f(x)=x^3$}{$ f(x)=\sqrt{x}$}{$f(x)=e^x-1 $}{$ f(x)=\ln\left(x+1\right)$}
%%%朝阳卷8
		\qs 在平面直角坐标系$xOy$中,已知点$ A\left(\sqrt{3},0\right) ,~B\left(1,2\right)$,动点$P$满足$ \vv{OP}=\lambda\vv{OA}+\mu\vv{OB} $,其中$ \lambda ,~\mu\in\left[0,1\right] ,~\lambda+\mu\in\left[1,2\right]$,则所有点$ P $构成的图形面积为\xx
		\onech{$ 1$}{$ 2$}{$ \sqrt{3}$}{$ 2\sqrt{3}$}
		\qs 设函数$f(x)=\sin\left(4x+\dfrac{\pi}{4}\right)\left(x\in\left[0,\dfrac{9\pi}{16}\right]\right)$,若函数$ y=f(x)+a\left(a\inR\right) $恰有三个零点$ x_1,\ x_2,\ x_3 \left(x_1<x_2<x_3\right)$,则$ x_1+2x_2+x_3 $的值是\xx
		\onech{$ \dfrac{\pi}{2}$}{$ \dfrac{3\pi}{4}$}{$ \dfrac{5\pi}{4}$}{$ \pi$}
		\qs 已知点$ M $在圆$ C_1: \left(x-1\right)^2+\left(y-1\right)^2=1$上,点$ N $在圆$ C_2:\left(x+1\right)^2+\left(y+1\right)^2 =1$上,则下列说法错误的是\xx
		\fourch{$ \vv{OM}\bm{\cdot}\vv{ON}$的取值范围是$ \left[-3-2\sqrt{2},~0\right] $}{$ \abs{\vv{OM}+\vv{ON}}$的取值范围是$ \left[0,~2\sqrt{2}\right] $}{$\abs{\vv{OM}-\vv{ON}}$的取值范围是$ \left[2\sqrt{2}-2,~2\sqrt{2}+2\right] $}{若$\vv{OM}=\lambda\vv{ON} $,则实数$ \lambda $的取值范围是$\left[-3-2\sqrt{2},~-3+2\sqrt{2}\right]$}		
		\qs 把$ 4 $件不同的产品排成一排,若其中的产品$ A $与产品$ B $都摆在产品$ C $的左侧, 则不同的摆法有\tk 种.(用数字作答)
		\qs 一次数学会议中,有五位教师来自$ A,~B,~C $三所学校,其中$ A $学校有$2$位,$ B $学校有$ 2 $位,$ C $学校有$ 1 $位.现在五位老师排成一排照相,若要求来自同一学校的老师不相邻,则共有\tk 种不同的站队方案.
		\qs 设函数$f(x)=\begin{dcases}
			x,&x\ge a\\
			x^3-3x,&x<a
		\end{dcases}.$\\
		\ding{192} 若函数$f(x)$有两个零点,则实数$ a $的取值范围是\tk;\\
		\ding{193} 若$ a\le -2 $,则满足$ f(x)+f(x-1)>-3 $的$ x $的取值范围是\tk.
		\qs 已知函数$f(x)=2\sqrt{3}\sin x\cos x+2\cos ^2x-1$
		\begin{parts}
			\part 求$ f\left(\dfrac{\pi}{6}\right) $的值;
			\part 求$f(x)$的单调递增区间.
		\end{parts}
	\qs 已知函数$f(x)=\dfrac{\ln x}{x+a}$.
	\begin{parts}
		\part 当$ a=0 $时,求函数$f(x)$的单调递增区间
		\part 当$ a>0 $时,若函数$f(x)$的最大值为$ \dfrac{1}{e^2} $,求$ a $的值.
	\end{parts}
	\qs 已知椭圆$C:\dfrac{x^2}{a^2}+\dfrac{y^2}{b^2}=1~(a>b>0)$的离心率为$ \dfrac{\sqrt{3}}{2} $,且点$ T\left(2,1\right) $在椭圆$ C $上,设与$ OT $平行的直线$l$与椭圆$ C $相交于$ P,~Q $两点,直线$ TP,~TQ $分别与$x$轴正半轴交于$ M,~N $两点.
	\begin{parts}
		\part 求椭圆$C$的标准方程;
		\part 判断$ \abs{OM}+\abs{ON} $的值是否为定值,并证明你的结论.
	\end{parts}



%%东城导数题目


	\end{questions}
\end{document}
